%\begin{savequote}[75mm] 
%Creativity is bred from restriction.
%\qauthor{Mark Rosewater} 
%\end{savequote}

\chapter{Introduction}
\label{ch:intro}
%\newthought{We did not enter our task unprepared.} From the outset of our project, we understood the problems inherent in attempting to marry the wildly different expectations of our %sponsors, NHP Electrical Engineering Products and the University of Melbourne.
%One party desired beauty; a streamlined and intuitive demonstration unit. The other, a demonstration of technical prowess and the application of mathematics to reality.

%This document outlines the decisions and compromises we made in an attempt to fulfil both expectations.

%From the beginning, we realised that there would need to be a series of modules that would make up the system. There would need to be a module allowing the expression of curves that the user would like drawn, a module converting the curves into a set of movement instructions and a module that transmits those instructions to the printer. On the PLC side we would need to create a module that would receive movement instructions, a module that would interpret those instructions and a module that would cause the motor drivers to act on those instructions. A high level description of the components of the system can be seen in Fig~\ref{fig:system}.


\section{Abstract}
The DoodleBot Project was a year long undertaking completed by Rahul Victor and Anthony Evans for the  Engineering Capstone Project subject at the University of Melbourne. The team produced a functional 2.1 axis CNC machine device capable of autonomously reproducing bitmap images or photos approximated by B-splines with a marker. The device incorporates open-loop time-optimal control that minimizes the time taken to draw the image. As an industry partner, NHP Electrical Engineering Products Pty Ltd proposed the project and provided a significant portion of the hardware for the project. This report will outline the project requirements, detail the engineered solution, justify the design and discuss performance and limitations. 

\section{Project Requirements}

A priority for the DoodleBot Project was to deliver to the requirements of two different stakeholder groups - the University of Melbourne and NHP Electrical Engineering Products Pty Ltd.

\subsection{Melbourne School of Engineering Requirements}
As part of the Bachelor of Engineering (Electrical) and Master of Engineering (Electrical) courses, students are required to complete ELEN90067 Electrical Engineering Capstone Project, a year long subject that involves completing an engineering project designed to develop and demonstrating practical engineering skills and theory. 

The expectations of the Melbourne School of Engineering were outlined in the available marking criteria and subject guidelines. The team is assessed based on several deliverables:
	\begin{itemize}
		\item A final technical report documenting the design, justification and performance of the project (this report)
		\item An oral presentation showcasing the design, justification and performance of the project
		\item A demonstration of the project at the annual Endeavour Exhibition to the general public and industry representatives
	\end{itemize}
and are marked to the following assessment criteria:
	\begin{itemize}
		\item Demonstration of technical skill and understanding of engineering concepts
		\item Understanding relevant literature in the field
		\item Quality of design and development
		\item Quality of implementation, experimentation/testing and results
		\item Quality of presentation (for each deliverable)
	\end{itemize}
	
\subsection{NHP Electrical Engineering Products Pty Ltd Requirements}
The DoodleBot is an industry partnered project with NHP Electrical Engineering Products Pty Ltd (henceforth in this report referred to as NHP). NHP's requirements for the project were agreed over several meetings, and detailed in a Project Charter produced by the DoodleBot Team and digitally signed off by the NHP supervisor.

The agreed deliverables, to be handed over at the conclusion of the year, are:
	\begin{itemize}
		\item A functional device (including all hardware and software)
		\item Two promotional videos (one with a technical focus and one with a marketing focus) showcasing the features of the device
	\end{itemize}	

The specifications of the finished product are:
	\begin{itemize}
		\item A CNC machine implemented on NHP specified hardware capable of 'drawing' single colour images to paper
		\item Capable of discrete 2 axis control
		\item Two state operation in third axis (on-off)
		\item Input incorporating image/photo recognition
		\item Software to be implemented on PLC and PC, with communication via a ethernet interface
		\item An intuitive user experience requiring no technical expertise
	\end{itemize}

\section{Project Scope/DoodleBot Team Contributions}
The responsibility of the DoodleBot team was to design, implement and construct the following components:
	\begin{itemize}
		\item Control problem formulation, solution design and software implementation. 
		\item Design and software implementation of image/photo input process
		\item Design and software implementation of user interface
		\item PC-PLC Network Interface design and software implementation
		\item Z-axis design and construction
		\item Wiring loom construction and assembly of components
	\end{itemize}
	
NHP's contribution to the project was to:
	\begin{itemize}
		\item Provide an assembled CNC frame with mechanical components for two axis movement;
		\item Provide a programmable logic controller (PLC), 2x stepper motors and 2x stepper motor PLC modules
		\item Provide adequate software and licenses for PLC programming
	\end{itemize}


\section{Performance Indicators}
In addition to the requirements, the DoodleBot Team identified several key performance indicators to design for:

	\begin{itemize}
		\item How accurately the mechanical device is able to reproduce the desired input
		\item How fast the mechanical device is able to draw the desired input
		\item How well the system is able to create a bicolour representation of the original image
	\end{itemize}