\subsection{MATLAB\textsuperscript{\textregistered} Processing}
The major mathematical processing aspects of the DoodleBot's URBS curve processing pipeline were all written in MATLAB\textsuperscript{\textregistered}. This was mostly due to the superior support that MATLAB\textsuperscript{\textregistered} offers programmers with regard to mathematical manipulation. Hence, the Java program needed to interface with MATLAB\textsuperscript{\textregistered} as a client, in order to perform processing of the user input. The MATLAB\textsuperscript{\textregistered} functions were designed to take a standard input and produce a standard output for each of the two uses that the Java program required; image processing and curve trajectory optimisation. In the case of the image processing function the Java program would provide the file system path of the image and MATLAB\textsuperscript{\textregistered} would return a series of URBS curves. These curves were added to the input buffer within the Java program. In the case of processing a curve, the Java program would supply MATLAB\textsuperscript{\textregistered} with a list of control points, plus other configuration information such as maximum accelerations for each axis and the number of path segments to use for the discrete approximation for optimisation. After the processing was complete, MATLAB\textsuperscript{\textregistered} would return two arrays of velocity profiles, one for each axis. The Java program would add these results to the server program thread for the DoodleBot to request and act upon. The interface between the Java program and MATLAB\textsuperscript{\textregistered} relied on the matlabcontrol Java API. The MATLAB\textsuperscript{\textregistered} interface program thread instantiates a MATLAB\textsuperscript{\textregistered} instance, to which it can send commands composed of text strings. The MATLAB\textsuperscript{\textregistered} interface thread composed strings and transferred control point data to the MATLAB\textsuperscript{\textregistered} instance and extracted the data after the processing was complete. This effectively wrapped the MATLAB\textsuperscript{\textregistered} functionality that was desired, allowing the Java program to access the advanced mathematics required to perform the optimisation and image processing.