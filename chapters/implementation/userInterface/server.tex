\subsection{Profile Server}
The output of the Java program is the point of hand-off of user commands from the user's terminal to the DoodleBot. These commands consist of sampled velocity profiles, which describe the speeds at which the DoodleBot's axes should be travelling at regular points in time. This results in the DoodleBot tracing out a particular curve. In order to communicate, the communication system simply serves commands upon the DoodleBot's communication module requests. The interface is entirely passive from the server's point of view. The server sends the DoodleBot commands upon receiving a request if it has commands to give. The general flow of the communication proceeds as follows. The DoodleBot requests a command header. If the server program thread has a command stored in its buffer, it will reply with a header containing the number of samples within the command, as well as other configuration information such as the maximum speed that the DoodleBot should travel. Once the DoodleBot has processed this initial information, it begins to request instructions from the server. Each of the instructions are numbered and the instructions are each requested specifically. The server will continue serving the requested instructions from the same command until the DoodleBot terminates the command by requesting another header. At this point, the server thread discards the current buffered command and will respond with the next header reply if there is another command stored in the buffer.
The server thread uses the Java.io and Java.net libraries for the ethernet interface features it provides. As it runs on a separate thread, it is autonomous from the remainder of the Java program and will continue to serve commands to the DoodleBot regardless of other processing taking place on the host system.