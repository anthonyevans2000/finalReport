O %ne of the design criteria for the DoodleBot was that the product was easy for a non-technical user to operate. An intuitive Graphical User Interface (GUI) is required which abstracted the URBS curve generation, image filtering and optimisation processing from the user. This would instead give the user an experience which was intuitive and similar to previous experiences with computer programs. To this end, a GUI similar in design to most drawing programs was developed.
%Given that the user would be interacting with a simple interface, the pipeline for the generation of printing instructions was automated. This was achieved with a Java program, modules of which formed the GUI, interface with MATLAB\textsuperscript{\textregistered} and the server to the DoodleBot. 
The Human Machine Interface is the module that allows a user to provide input to the DoodleBot System. One of the requirements specified in Chapter ~\ref{ch:intro} was that the system provides an intuitive user experience that requires no technical expertise and so the Human Machine Interface involves a Graphical User Interface (GUI) designed towards this purpose.

The user is able to provide input either through image files or by manually drawing an URBS. The method for drawing URBS aimed to replicate behaviour used in most standard vector graphics software to provide something that requires less of a learning curve to use.

Once the user has provided the input, the remaining steps in the process in the system are completely automated and the HMI handles the data flow through other modules implemented on the PC side including: image to spline interpolation, optimisation and input to the network interface server. This was achieved with a Java program with modules that formed the GUI, interface with MATLAB\textsuperscript{\textregistered} and the network interface server to the PLC.