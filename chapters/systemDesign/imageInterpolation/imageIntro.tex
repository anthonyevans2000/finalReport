A desired feature that was identified early in the conception of the Doodlebot was a method of easily generating splines. This would relieve the user from the load of generating a large number of splines from scratch. The input mechanism that made the most sense was an image input from a device like a digital camera. With this design goal in mind our aim was to turn a digital image into a representative set of splines.

The largest issue with such an endeavour lies in balancing the retention of qualitative detail in an image with the number of splines which are eventually produced. Experiments with differing filters and edge extraction algorithms were undertaken until a suitable scheme was found. The following section documents these choices