\subsection{Linear and Convex Optimisation Approach}
The method used to solve the problem involves breaking the curve down into a discrete number of sections and solving the continuous problem in a piecewise manner, which allows for the restrictions to be more easily considered.

The problem is now to apply a numerical solving method in order to find the approximated optimum. Using the technique outlined in \cite{Schutter09}, the problem was reformulated such that the non-linear constraint $\ddot{\textbf{q}}(t) = \textbf{q}'(t)\ddot{s}(t) + \textbf{q}''(t)\dot{s}^2(t)$ becomes linear for the solution variables, whilst the cost function remains convex. Enforcing linearity in the solution variables allows for the solution to be found by efficient solvers, as well as ensuring that a solver will not be able to get caught in a local minima.
The key to this linearisation is representing the non-linear portion of the constraint $\dot{s}^2(t)$ with the solution variable $b(s)$ as so;
\begin{align*}
a(s) = \ddot{s}(s)\\
b(s) = \dot{s}^2(s)
\end{align*}
Such that the non-linear constraint becomes
\begin{align*}
\ddot{\textbf{q}}(s) = \textbf{q}'(s)a(s) + \textbf{q}''(s)b(s)
\end{align*}
Note that the time dependence of the path parametrisation is neglected. This is because the cost function is reformulated with a change of variables such that its dependence on $t$ is replaced by a dependence on $s$. Thus the solution that minimises the cost function will minimise the time taken implicitly without direct reference to time as shown here;

\begin{align*}
J(T) &= \int_0^T1\;dt\\
J(s(\cdot)) &= \int_{s(0)}^{s(T)} \frac{1}{\dot{s}}ds\\
	&= \int_0^1\frac{1}{\dot{s}}ds
\end{align*}

The cost function is trying to minimise the inverse of $\dot{s}(t)$, which is the equivalent of attempting to maximise the path velocity. Given a solution variable is $b(s) = \dot{s}^2(s)$, the cost can be formulated directly in terms of our solution variable as;
\begin{align*}
J(s(\cdot)) = \int_0^1\frac{1}{\sqrt{b(s)}}ds
\end{align*}

The change of variables gives us a cost that remains convex and is in terms of one of the solution variables.