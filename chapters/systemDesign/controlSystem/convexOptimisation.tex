\subsection{Convex Optimisation Approach}
The method we used to solve the problem involved breaking the curve down into a discrete number of sections and solving the continuous problem in a piecewise manner via a non linear solver, which allows us to take into consideration the restrictions imposed.
The issue with using a numerical solving method is the lack of a guarantee that a descent method will result in the global optimum. Using the technique outlined in \cite{Schutter09}, we reformulated the problem such that the non-linear constraint $\ddot{\textbf{q}}(t) = \textbf{q}'(t)\ddot{s}(t) + \textbf{q}''(t)\dot{s}^2(t)$ becomes linear in the solution variables, whilst the cost function  $ \min \; \int_0^T1\;dt$ remains convex.
The key to this is solving the path constraints in terms of the path parametrisation and creating the solution variables
\begin{align*}
a(s) = \ddot{s}(s)\\
b(s) = \dot{s}^2(s)
\end{align*}
Such that the non-linear constraint becomes
\begin{align*}
\ddot{\textbf{q}}(s) = \textbf{q}'(s)a(s) + \textbf{q}''(s)b(s)
\end{align*}
And is hence linear in the solution variables.

We reformulate the cost integral with a change of variables, in order to get the cost to be convex in terms of $b(s)$;
\begin{align*}
\int_0^T1\;dt &= \int_{s(0)}^{s(T)} \frac{1}{\dot{s}}ds\\
	&= \int_0^1\frac{1}{\dot{s}}ds\\
	&= \int_0^1\frac{1}{\sqrt{b(s)}}ds
\end{align*}

The results of this alteration to our problem are far reaching. Firstly, it can be seen that our problem is now linear in its constraints. This results in our viable solution space taking up a contiguous range, meaning that for the solver no isolated solutions are unreachable.
Secondly, the cost integral that integrates over the trajectory $s$ is convex in the solution variable $b(s)$, so that a solver will always be able to determine the correct direction to move the state such that it converges to the local optimum.
With the combination of these two properties, it can be seen that a solver can be employed to correctly solve for the optimal trajectory $\dot{s}^*(s)$, through the solution of $b^*(s)$

In order to form a discrete problem that a computer can iterate over and solve, the path is discretised and the trajectory is solved over each path segment. As the control input, $a(s)$ is given as being piecewise constant for each discrete segment of $s$. Thus, as it is finite and constrained by the system constraints. $\sqrt{b(s)}$ is piecewise linear and continuous, meaning $b(s)$ is piecewise non-linear and continuous. 