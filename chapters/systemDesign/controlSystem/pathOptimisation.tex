\subsection{Path Velocity Optimisation}
A machine that follows prescribed paths has an inherent trade-off between obeying the physical restrictions of the device and following a path at the fastest possible speed. There is a wealth of information surrounding the topic of the optimisation of such physical systems, seen in \cite{Bardine10},  \cite{Chen07}, \cite{Choi01}, \cite{Tseng09}, \cite{Schutter09} and \cite{Yeh99}. Each study was relevant to the optimisation that we had in mind, but used differing strategies and overcame different limitations to achieve an optimal result. The method that we decided was the most relevant to our problem and was the process that we ended up implementing was covered in \cite{Schutter09}. A comparison of the different optimisation strategies is instructive and can be seen in Appendix B.

The physics model we chose to construct limitations on our plant was a pair of double integrators. Given that we can determine the limits on $\ddot{\textbf{q}}(t)  = \begin{bmatrix}
\ddot{x}(t)\\
\ddot{y}(t)
\end{bmatrix}$, we can construct an equality that will impose limits on the movement of each of the two axes, via the common parameter $s$.
\begin{align*}
\textbf{q}(s) &= \sum^N_{i=1}\textbf{N}^2_iW_{i,2}(s) = \begin{bmatrix}
x(s)\\
y(s)
\end{bmatrix}\\
\textbf{q}(t) &= \textbf{q}\left(s(t)\right)\\
\dot{\textbf{q}}(t) &= \frac{d\textbf{q}}{ds} \frac{ds}{dt}\\
\ddot{\textbf{q}}(t) &= \frac{d}{dt}\left(\frac{d\textbf{q}}{ds}\right)\frac{ds}{dt} +  \frac{d\textbf{q}}{ds}\frac{d^2s}{dt^2}\\
 &= \frac{d^2\textbf{q}}{ds^2}\left(\frac{ds}{dt}\right)^2 +  \frac{d\textbf{q}}{ds}\frac{d^2s}{dt^2}\\
 &= \textbf{q}''(t)\dot{s}^2(t) + \textbf{q}'(t)\ddot{s}(t)
\end{align*}
Where $\textbf{q}''(t)$ and $\textbf{q}'(t)$ equal $\frac{d^2q(s(t))}{ds^2}$ and $\frac{dq(s(t))}{ds}$ respectively.

Our problem is to choose $\dot{s}(t)$ such that $\ddot{\textbf{q}}_{min} \leq \ddot{\textbf{q}}(t) \leq \ddot{\textbf{q}}_{max}$.
Initially, our problem seems like it could be a simple exercise in maximising $\dot{s}(t)$ and setting $\frac{d^2s}{dt^2} = 0$ such that we achieve the inequality
\begin{align*}
\ddot{\textbf{q}}_{min}& \leq \frac{d^2\textbf{q}(s)}{ds^2}\left(\frac{ds}{dt}\right)^2 \leq \ddot{\textbf{q}}_{max}
\end{align*}

However, this incorrect thinking outlines the crux of the problem that we need to solve. In order to alter $\dot{s}(t)$  to accommodate changes in $\frac{d^2\textbf{q}}{ds^2}$ along the curve, we must set a non-zero $\frac{d^2s}{dt^2}$. This can result in a naive algorithm that attempts to process the curve and maximise $\dot{s}(t)$ becoming trapped in a situation where it needs to have been previously decelerating along the path in order to meet the requirements of a section of the curve. The game is one where these key deceleration points need to be identified, as well as identification of the points where the system should accelerate along the path so as to minimise completion time.
For each point in the curve, one of the axes will be limiting the maximum $\dot{s}(t)$. This is the nature of bang-bang time optimal control and this result falls out when one tries to minimise the Hamiltonian, as will be seen in the next section.
In order to solve the problem of obtaining the minimum time of traversal whilst remaining within the constraints imposed by following the curve and torque limitations, a few options are available.

One could attempt to determine the limiting axis for each point in the curve then determine the optimal way of accelerating and decelerating the path velocity via iterative shooting methods to meet the imposed limitation.

Another option would be to break the path into discrete segments, calculate the maximum $\dot{s}(t)$ for each segment and accelerate along the path such that these maximum values are satisfied at the commencement of each of these discrete divisions along the path.

Each of these options were explored during our research and an equivalent implementation is discussed over the following sections.

Once the limiting axis and all switching points are identified, we effectively have the optimal path trajectory $s^*(t)$, from which we can calculate the desired velocities and positions along $x^*(t)$ and $y^	*(t)$ with ease.

Formally, the problem we need to solve can be expressed by;
\begin{align*}
\min_{T, \; s(\cdot), \; \ddot{\textbf{q}}(\cdot)} \;& \int_0^T1dt\\
\text{subject to} \quad \ddot{\textbf{q}}(t) &= \textbf{q}'(t)\ddot{s}(t) + \textbf{q}''(t)\dot{s}^2(t)\\
s(0) &= 0\\
s(T) &= 1\\
\dot{s}(0) &= 0\\
\dot{s}(T) &= 0\\
\dot{s}(t) &\geq 0\\
\ddot{\textbf{q}}_{min} &\leq \ddot{\textbf{q}}(t) \leq \ddot{\textbf{q}}_{max}\\
\text{for } t &\in [0,T]\\
\end{align*}