\subsubsection{SeDuMi}

The discretised optimisation problem is formulated into the second order cone solver SeDuMi so as to be solved with numerical efficiency. An automated function programs the constraints on each path segment. The bounds on $b(s)$ and $a(s)$ are defined with respect to the acceleration constraints. The calculation of the maximum $\textbf{q}''(s)$ and $\textbf{q}'(s)$ for each discretised path section gives the variables defining these acceleration constraints. The starting and finishing values of $b(s)$ are anchored to zero. Finally, two solution variables are introduced in order to enable the efficiency of a second order cone program. For each discrete section, $c(s)$ is introduced as $\sqrt{b(s)}$ and $d(s)$ is introduced as the incrementing value of the cost integral.
The entire solving functionality was automated and wrapped in a function that could solve the optimal trajectory for a curve  given simple arguments of an URBS representation and the desired torque constraints.
