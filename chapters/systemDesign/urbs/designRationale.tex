\subsection{Design Rationale}


The decision to use a method of restricting and structuring the input curve type was made for several reasons. 

Firstly the structure that a spline based representation gives to the input data gives natural compression for that data, as compared to storing the location of each individual point upon the desired curve. This allows for easy storage and manipulation of input curves by the GUI program. 

Secondly the restriction of the curves to splines of the 3rd order ensures that the parametrised curve $\textbf{q}(s)$ has the property of being twice differentiable;
\begin{align*}
\textbf{q}(s) \in \mathbb{C}^2
\end{align*}
If this were not the case, the path velocity could not be be guaranteed to be continuous. The plant would then require infinite acceleration at those discontinuities in order to travel the path at any speed. 

Finally the parametrisation of the curve $\textbf{q}(s)$, allows us to easily retrieve the path velocity $\textbf{q}'(s) = \frac{dq(s)}{ds}$ and path acceleration $\textbf{q}''(s) =  \frac{d^2q(s)}{ds^2}$ along any point on the curve. This is useful in calculating the force along an axis required for any given $\dot{s}(t)$

